\chapter*{Introduction}
\addcontentsline{toc}{chapter}{Introduction}
\markboth{Introduction}{}


% The history of topological materials is just a little over thirty years old.
% A good point to start is the discovery of the quantized Hall conductance in two-dimensional semiconductor samples by von Klitzing in the early 1980s~\cite{Klitzing1980,Klitzing1992}.
% He found that the Hall conductance develops plateaus as a function of the magnetic field which are exactly quantized in multiples of a fundamental constant that depends on the elementary charge and Planck's constant.
% In particular, it is independent of any material properties or external conditions.
% Due to the high precision of the quantization levels, for which an explanation was given in the following years by Laughlin and Halperin~\cite{Laughlin1981,Halperin1982}, this effect immediately found applications in metrology as a direct measurement of the fine structure constant and as a standard for the unit of resistance.
% The discovery by von Klitzing was awarded with the 1985 Nobel Prize in physics.

% A few years after the discovery, Thouless and others discovered the first connection to topological properties~\cite{Thouless1982,Niu1985,Kohmoto1985,Avron1985,Kohmoto1989,Bellissard1994,Avron2003}.
% They found a direct relation between the Hall conductance and a topological invariant called Chern number.
% In much the same way that the number of `handles' of a closed two-dimensional manifold can be calculated by an integration over its curvature, the Chern number of a Hamiltonian can be calculated by integrating its Berry curvature over a periodic two-dimensional configuration space.
% Similar to the Gaussian curvature of the manifold, the Berry curvature of the quantum mechanical system quantifies the geometric changes of the wave functions under transport around closed loops~\cite{Berry1984,Zak1989}.
% The connection of the quantized Hall conductance to a topological invariant manifests itself in the robustness of the physical effect against local perturbations.

% A related, but considerably more complex phenomenon was experimentally discovered by Tsui, Störmer and Gossard in 1982 at even lower temperatures in cleaner samples~\cite{Tsui1982}.
% They found that the Hall conductance could additionally develop plateaus at certain fractional values of the filling factor, the ratio between the number of electrons and the number of magnetic flux quanta threading through the sample.
% These plateaus correspond to fractionally filled Landau levels and could not be explained by a single-particle treatment.
% Once again, it was Laughlin who was able to explain the phenomenon~\cite{Laughlin1983}, winning him the 1998 Nobel Prize in physics together with Tsui and Störmer.
% He found that the two-dimensional electron gas condenses into a new state of matter, a quantum fluid with fractionally charged excitations and anyonic statistics.
% This strongly correlated state of matter is an example of a topologically ordered state with a ground state degeneracy that depends on the topology of the underlying space and a robustness against local perturbations~\cite{Wen1990,Wen1995}.
% The structure of some fractional quantum Hall states still remains unexplained.
% The most prominent example is the even-denominator state at a filling of $\sfrac{5}{2}$ that was experimentally observed as early as 1987 by Willet \etal~\cite{Willett1987}.
% Particular interest in this state draws from work by Moore and Read~\cite{Moore1991}, suggesting that it might give rise to quasiparticles with non-Abelian statistics.
% Interchange of non-Abelian anyons leads to a change in the ground state manifold of the system. This property can be utilized for fault-tolerant quantum computation, an idea that has been proposed by Kitaev in 1997~\cite{Kitaev2003}

% Fundamental questions about the nature of these states as well as their prospective use in topological quantum computation spur the research in this field today.
% Traditional experiments with semiconductor samples remain challenging due to immense requirements on the sample quality, low temperatures and high magnetic fields.
% With the turn of the century and the advent of ultracold gases experiments, new ideas how to reach the Quantum Hall regime emerged.
% Unmatched control over system parameters as well as the ability to manipulate and observe on the single-particle level turn these systems into an optimal platform to advance our understanding in the field of Quantum Hall physics.
% A fundamental problem appears when trying to emulate the effect of the magnetic field.
% Electrically neutral atoms clearly do not couple to the magnetic vector potential in the way that electrons do.
% Various solutions to this problem have been proposed and experimentally implemented.
% Following an analogy that goes back to ideas by Larmor around 1900, it is possible to use a rapid rotation to induce an effective magnetic field for the neutral particles~\cite{Larmor1900}.
% In the two-dimensional system, the frequency of rotation corresponds to the effective magnetic field strength parametrized by the cyclotron frequency.
% Likewise, the Coriolis force is in one-to-one correspondence with the Lorentz force.
% Starting in the early 2000s, experiments in this respect have advanced over the years~\cite{Schweikhard2004,Bretin2004,Cooper2008,Fetter2009}.

% An alternative route was followed by Haldane~\cite{Haldane1988}.
% In 1988, he proposed a lattice model with broken time-reversal symmetry which showed a quantum Hall effect without the requirement of Landau levels that would be generated by an external magnetic field.
% The Haldane model utilizes complex tunneling phases that respect the symmetry of the lattice and generate a topological band structure.
% It is a showcase for a class of materials called Chern insulators.
% They behave similar to ordinary band insulators, but have conducting states at the edge of the material: a physical manifestation of their non-trivial Chern number~\cite{Hatsugai1993}.
% For charged particles, the required complex tunneling phases are connected to the external magnetic field through a Peierls substitution~\cite{Peierls1933}.
% In this regard, synthetic magnetic fields can be created for neutral particles by realizing complex tunneling phases.
% Powerful approaches are optical flux lattices~\cite{Cooper2011}, laser-assisted tunneling~\cite{Aidelsburger2011,Aidelsburger2013,Miyake2013,Kennedy2015} or lattice shaking methods~\cite{Struck2012,Struck2013}.
% The latter has recently been used by Jotzu \etal to realize the Haldane `toy model' with ultracold fermions in an optical lattice~\cite{Jotzu2014}.

% Finally, another strategy is to use spin-orbit coupling techniques~\cite{Lin2011,Cheuk2012,Wang2012,Hamner2014,Jimenez-Garcia2015} to realize topological phases.
% The interplay between external and internal degrees of freedom can lead to phenomena which are similar to the magnetic field counterparts.
% In 2005, Kane and Mele showed that spin-orbit coupled electrons in graphene can realize a topological system which encapsulates two time-reversed copies of Haldane's model~\cite{Kane2005a,Kane2005}.
% The resulting arrangement is an example for a time-reversal invariant topological insulator. It shows a quantum spin Hall effect where the two spin-components have a Hall conductance with opposite sign~\cite{Qi2011,Hasan2010}.
% A physical realization in semiconductor quantum wells was proposed by Bernevig \etal in 2006~\cite{Bernevig2006a,Bernevig2006b} and experimentally demonstrated by König \etal one year later~\cite{Konig2007}.

% A variety of experimental methods to probe topological materials have been established in recent years.
% Edge states have been observed in different systems like silicon photonics~\cite{Hafezi2011a,Hafezi2013}, photonic lattices~\cite{Rechtsman2013} and phononic mechanical systems~\cite{Susstrunk2015}.
% Furthermore, the perfect control over ultracold atomic systems has led to new ways to directly measure topological properties like the Zak phase~\cite{Atala2013}, the Berry curvature~\cite{Duca2014} or the Chern number~\cite{Aidelsburger2014}.

\textcolor{red}{Include a thermopower review, i.e. small recount of thermal responses measurements in Corbino and Hall bars. Cite here works and make sure to include Molenkamp, Benenti, Koba, Zalinge papers. }

The quantum Hall effect (QHE) \cite{klitzing1980new} is one of the cornerstones to electrical metrology and the redefinition of units of 2019 \cite{CGPM26}. It was already in the hart of the adendum of the 1990 revision of the SI, then, the electrical metrology comunity was alowed to use quantum standards kind of outside the SI. Its link being the calculable capacitor. There are many advisable works regarding this point like \cite{Valdes2019}
, and a versatile playground for solid state physics since its discovery more than 40 years ago \cite{klitzing1980new}.

As mentioned before this thesis original focus was on the study, development and implementation of methods to determine the temperature of systems under the quantum Hall (QH) state. Furthermore, it also covered possible new thermal effects under this fascinating system. Regarding the latter, we were able to grasp a promissing cooling behavior, that will be presented in \textcolor{tmagenta}{citar acá el capítulo de cooling}.

Studying the thermopower effects under the QH regime is a particular difficult task, and caught the attention of many works in the past \cite{chickering2010,chickering2010thermopower,chickering2013,kobayakawa2013diffusion} \textcolor{tmagenta}{citar los trabajos que midieron themopower antes}. But the results usually presented some issues regarding consistency and interpretation, this was pointed out by Y. Barlas and K. Yang \cite{Barlas2012}. They showed that the use of devices with a Hall-bar geometry presented intrinsic problems because of the voltage component mixing, resulting in a misalignment between the voltage and temperature gradients in the devices. They also presented an elegant solution, by proposing a Corbino (ring shaped) geometry, that imposes the gradients to be parallel. When working on this problem their focus was on the study of new possible non-abelian physics in the $ 5/2 $ fractional state. This was experimentally studied, in Hall-bar geometry, by the 
Eisenstein group \cite{chickering2010,chickering2013}, the Molenkamp group \textcolor{tmagenta}{citar molenkamp y otros trabajos en esta dirección}. But our focus is in a different direction, we exploited this work to produce new methods and approach to determine temperature gradients, thermal properties and possible new applicable effects of the QH states. 



 Two very interesting side of the effects are its thermal and thermoelectric properties, for which many insightful works have been produced \cite{chickering2010thermopower,chickering2013thermoelectric,VanHouten1992,Zalinge2003,endo2019spatial,Liu2018} \textcolor{red}{citar trabajos de chickering, molenkamp, von klitzing etc sobre estos estudios}. 

In this kind of studies to measure the 2D electron system (2DES) temperature or temperature gradient is of main importance. But many times one relies on a typical resistance transducer, that, depending on the kind of measurement, might be not the best suited configuration for the measurement in hand.


Thermoelectric voltages are a consequence of a temperature gradient along the electronic system. 
In a traditional experiment the usual configuration is a hall bar \cite{chickering2010thermopower,chickering2013thermoelectric} \textcolor{red}{citar aquellos con hall bar}, where one end of a rectangular sample is clamped to a thermal bath while the temperature at the other is increased by means of some kind of heating system, leading to a temperature distribution along the length of the sample. This configuration presents some issues, because it mixes longitudinal and transverse responses as described by Barlas and Young in \cite{barlas2012thermopower}. They show that a Corbino device averts the problem, for which its geometry imposes a parallel thermal gradient to the electrical potential drop (thermopower). Some works have taken advantage of this geometry and its possibilities \cite{Zalinge2003,kobayakawa2013diffusion,real2020thermoelectricity,mateos2021thermoelectric}. 
However, a thermal gradient cannot be straightforward applied in a ring-shaped Corbino structure where the center of the ring should be heated and the rim be cooled. 
It turns out to be difficult to connect the outer rim of the Corbino to a thermal bath with a vanishing thermal resistance. 
Another issue being the high thermal conductivity of the substrate which leads to small thermal gradients at reasonable heater powers. \textcolor{blue}{In a different approach van Zalinge et al. \cite{Zalinge2003} used illuminated samples, a laser acting as a hotspot probe and using it to study anisotropies in the photo and thermovoltages. They found anisotropies regarding the thermovoltage responese to the different main directions of the crystal.} Kobayakawa et. al\cite{kobayakawa2013diffusion} used a Corbino configuration having its center connected to a thermal sink while a circular rf heater surrounded the ring, and focus in low mobility and low magnetic fields. 

In this work we use a more conventional approach, which consists in placing  a central resistive heater and gluing a rim of the sample to the chip carrier, thermally connected to the cold finger of a $^3$He cryostat. Taking advantage of the temperature dependence of the conductance minima in the quantum Hall (QH) regime we experimentally estimate the thermal gradient produced along the sample for different heater powers and base cryostat temperatures. We demonstrate that there exists a linear relation between the power applied to the heater and the temperature gradient in the sample. The good agreement found suggests a potential application of the device for sensing small temperature biases in micro and nano devices. The Corbino device we study is schematically showed in Fig.~\ref{fig:fig1-expSetup}.